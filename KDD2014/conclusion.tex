\section{Conclusion}
Summarizing this work, we introduced the first solution that is
explicitly able to find meaningful overlapping in the graph structure
as well as in its respective attribute subspace. We outlined the importance
of these overlapping communities especially for attributed graphs and the information
gain that can be received by it.
Our method IROC applied the concept of information theoretic measures like entropy
and an Minimum Description length (MDL) formula designed for this callenge
 to elegantly avoid a) redundancy in the attribute space as well as in the network itself and
 b) the need to set in an unsupervised setting typically unknown input parameters to achieve good results.
Our experiments clearly showed that IROC is able to outperform all relevant comparison methods on synthetic data and on real world data of social networks.
