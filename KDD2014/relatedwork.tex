\section{Related Work and Discussion}
Our proposed algorithm IROC is design for clustering the attributed graph under the permission of overlapping in both vertices and the subspace of attributes. The related work therefore comprises two parts: attributed graph clustering and overlapping community detection methods.

\subsection{Attributed Graph Clustering}
Attributed graph is an extension of general graph by involving the attributes of the vertices. The key point of mining such complex data is to combine structural connections and characteristics of the vertices. Paper \cite{DBLP:journals/pvldb/ZhouCY09} augments graphs by considering each attribute as vertex. A Random walk is utilized on the augmented graph so as to create a unified similarity measure which combines structural and attribute information. However, the algorithm is carried out in a K-Medoids framework and partitions the attributed graph. It also needs to input the number of clusters and parameters for the random walk, such as number of steps and the probability to go back. Moreover, paper \cite{DBLP:conf/icdm/ZhouCY10} proposes an incremental algorithm to improve efficiency of \cite{DBLP:journals/pvldb/ZhouCY09}. Paper \cite{DBLP:conf/sigmod/XuKWCC12} proposes an algorithm named BAGC based on Bayesian probabilistic model to partition attributed graphs. Structural and attributed information are fused by the probabilistic model and clustering problem is transferred to a probabilistic inference problem. Similarly, the algorithm needs to input the number of clusters, and many other parameters to construct the necessary probability distribution. Obviously, these partition based methods can not detect any overlapping of clusters. And they do not find coherent attributed subspace of cluster. For these partitioning approaches, we choose BAGC as a comparison method. 

Guennemann et al. proposes an algorithm named DB-CSC (Density-Based Combined Subspace Clustering) \cite{DBLP:conf/pkdd/GunnemannBS11} which is based on the classical density-based clustering algorithm DBSCAN \cite{DBLP:conf/kdd/EsterKSX96}. The new proposed DB-CSC inherits the advantages of DBSCAN which can detect clusters of arbitrary shapes and sizes. It defines a combined local neighborhood by finding vertices, which belong to the intersects of $k$-neighborhood of vertices and $\epsilon$-neighborhood of subspace of the attributes. Based on the new defined neighborhood, some density related properties like high local density, local connected and maximality is defined, thus the fulfilled clusters can be detected. Instead of giving the number of clusters, DB-CSC needs parameters like $\epsilon$-neighborhood, $k$-neighborhood and a minimum number Minpts. In order to remove redundant clusters of the above two algorithms, the authors proposed a definition which is used to calculate redundancy between clusters. After adopting the strategy of removing redundancy clusters, the combined new algorithms need more parameters, $robj$ and $rdim$ which measure how much overlap between clusters is allowed without redundancy. Therefore, DB-CSC needs an isolate process and set several parameters to remove redundancy. If the parameters is set unpropertied, redundancy still exists. In comparison, IROC obtains the non-redundant results without setting any parameters. Moreover, as mentioned in experiments part, judging from the distance defined in the paper, DB-CSC is defined to deal with the attributed graphs with numerical attributes. But categorical attributes like gender, hobby .etc are common in social network data set. Therefore, IROC is able to detect overlapping clusters of the categorical attributed graph and meanwhile find the coherent attribute subspace of each cluster.    


%Guennemann et al. propose Gamer\cite{DBLP:conf/icdm/GunnemannFBS10} and DB-CSC\cite{DBLP:conf/pkdd/GunnemannBS11}, which combines dense subgraph mining with subspace clustering. In experimental section, we choose the latest DB-CSC as a comparison method. In DB-CSC, a new neighborhood which combines structure and attribute information is defined to fit the density based clustering implemented on graph data with additional feature vectors. Clusters with overlapping can be detected, meanwhile corresponding subspace can be discovered by depth first searching. However, density based algorithm need to predefine many parameters and many clusters are produced as well as redundancy. Most importantly, DB-CSC is restricted to some types of overlap: Within the same subspace clusters are not allowed to overlap in terms of the objects. Overlap is only possible for graph clusters in different subspaces. Our proposed algorithm combine structural and attributed information based on information theoretic ideas and is not restricted to some specific types of overlap. All possible combinations of overlap in terms of objects and subspaces are in principle possible but only reported as a result if they pay-off with respect to data compression. IROC is the first method balancing quality and redundancy in an information-theoretic way.
 
PICS\cite{DBLP:conf/sdm/AkogluTMF12} is also a parameter free algorithm based on MDL principle. It is able to mine cohesive clusters from an attributed graph with similar connectivity patterns and homogeneous attributes. However, it can not detect any overlapping and it cluster the vertices and attributes separately. Thus it is hard to find out which subspace belongs to which clusters. Additionally, Sun et al. \cite{DBLP:journals/pvldb/SunAH12} proposes a model-based method to clustering heterogeneous information networks which are containing incomplete attributes and multiple link relations. Also marginally related to our method are the approaches \cite{DBLP:journals/pvldb/SilvaMZ12}\cite{DBLP:conf/sdm/MoserCRE09}\cite{DBLP:conf/kdd/TongFGE07} achieving numerous small cohesive subgraphs, which aim to discover a correlation between node attributes and small subgraphs. Paper \cite{DBLP:conf/sigmod/TianHP08} summarizes multi-relational attributed graphs by aggregating nodes by using selected attributes and relations.

\subsection{Detecting Overlapping Communities}
The key point of acquiring overlapping clusters is how to assign a vertex to multiple labels. In first instance, \cite{nature} reveals overlapping phenomena of complex networks in nature, and achieves overlapping communities by seeking k-cliques which contains overlapping vertices. Paper \cite{DBLP:conf/kdd/ZhangY12} proposes an algorithm based on matrix factorization. The assignment of each vertices is stored as probability in a matrix with a number of dimensions equal to the number of the community. By fuzzy allocation, overlap between communities is achieved. CONGA\cite{DBLP:conf/pkdd/Gregory07} proposed by Gregory is an algorithm which aims to detect overlapping communities by iteratively calculating two betweenness centrality based concepts ``edge betweenness" and ``split betweenness" of all edges and vertices respectively and removing the edge or splitting the vertex with the highest value until no edges remain. As betweenness centrality is a global measure of vertices in a graph, the calculation of the two concepts depends on counting the number of shortest paths of all pairs of vertices, which is really time consuming. In order to speed up the algorithm CONGA, the author proposes an algorithm named CONGO\cite{DBLP:conf/pkdd/Gregory08} by calculating local betweenness instead of global betweenness. In the new algorithm, a parameter $h$ is added, which is a threshold that the shortest path which is more than h is ignored. Thus the concepts only need to recalculate locally to save time complexity. Both CONGA and CONGO need user to predetermined the number of clusters $k$. In this paper, we compare IROC with CONGO to prove the efficiency of our algorithm to detect overlapping clusters. Actually, the overlapping community detection algorithms which are mentioned above are all only considering the structural information of graph with no additional attributes.


% Gregory proposes an algorithm named CONGA\cite{DBLP:conf/pkdd/Gregory07}, which is based on measuring betweenness centrality. By continuous splitting the vertex with the highest betweenness value, the communities are formed and meanwhile the vertex is assigned to multiple communities. Then a fast algorithm\cite{DBLP:conf/pkdd/Gregory08} is proposed for speeding up CONGA by calculating a local betweenness value. Coscia et al.\cite{DBLP:conf/kdd/CosciaRGP12} proposes an algorithm by extracting redundancy subgraphs by defining graph operations, label propagation algorithm performance on these subgraphs while merging them to obtain the overlapping communities. Similarly, our algorithm is achieving its overlapping community by mining redundancy of subgraphs. However, comparing with our algorithm, this algorithm outputs too much clusters with large redundancy. 




